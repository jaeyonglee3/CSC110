\documentclass[11pt]{article}
\usepackage{amsmath}
\usepackage{amsthm}
\usepackage{amsfonts}
\usepackage[utf8]{inputenc}
\usepackage[margin=0.75in]{geometry}

\title{CSC110 Fall 2022 Assignment 2: Logic, Constraints, and Wordle!}
\author{Jaeyong Lee}
\date{\today}

\begin{document}
\maketitle

\section*{Part 1: Conditional Execution}

Complete this part in the provided \texttt{a2\_part1\_q1\_q2.py} and \texttt{a2\_part1\_q3.py} starter files.
Do \textbf{not} include your solutions in this file.

\section*{Part 2: Proof and Algorithms, Greatest Common Divisor edition}

\begin{enumerate}
\item[1.]

This can be explained using the following property of division: $\forall n, d\in \mathbb{Z}^{+}, d|n \Rightarrow d \le n$

According to this property of division, if d divides n, then d is lesser than or equal to n. From the precondition of the function, we know that m is lesser than or equal to n. Since d must divide both n and m to be considered a common divisor, we can conclude that the possible divisors must only go from 1 to the lower of the two integers m and n, which is m in the case of this function. 

\item[2.]

This can be explained using the following property of division: $\forall n\in \mathbb{Z}, 1|n$. Essentially, what this property tells us is that for every integer n, 1 divides n. Therefore, the set common\_divisors will never be empty since 1 is a common divisor of every integer, meaning we don't have to check for a case in which common\_divisors is empty before calling max(common\_divisors). 

\item[3.]

\begin{proof}
$\forall n, m, d\in \mathbb{Z}, d|m \land m \ne 0 \Rightarrow (d|n \iff d|n \mathbin{\%} m)$

Suppose $d|m \land m \ne 0$. We wish to show that $(d|n \iff d|n \mathbin{\%} m)$. Since this is a bi-conditional, we will split the proof into two separate parts. \\

Part 1: $d|n \Rightarrow d|n \mathbin{\%} m$

Assume $d|n$. We wish to show that $d|n \mathbin{\%} m$. We define $n \mathbin{\%} m$ to be equal to the unique integer r that satisfies $0 \le r < |m|$ and $\exists q\in \mathbb{Z}, n = qm + r$. From this, we know that $n = qm + n \mathbin{\%} m$, which can be rearranged to $n - qm = n \mathbin{\%} m$.\\

Writing this as $n \mathbin{\%} m = n + (-qm)$ shows that we can use the given property $\forall n, m, d, a, b\in \mathbb{Z}, d|n \land d|m \Rightarrow d|(an+bm)$ to show that $n \mathbin{\%} m = an + bm$ when we let a = 1 and b = -q.\\

From the given property, we know that $d|(an+bm)$ when $d|n$ and $d|m$. We've assumed $d|n$ and $d|m$ to be true in our proof. We have shown $d|(an+bm)$ to be equivalent to $d|n \mathbin{\%} m$. Therefore, we have proven $d|n \Rightarrow d|n \mathbin{\%} m$. \\ 

Part 2: $d|n \mathbin{\%} m \Rightarrow d|n$

Assume $d|n \mathbin{\%} m$. We want to show that $d|n$. \\
$d|n$ can be expressed as $\exists k_1 \in \mathbb{Z}, n = dk_{1}$ \\
$d|n \mathbin{\%} m$ can be expressed as $\exists k_2 \in \mathbb{Z}$, $n \mathbin{\%} m = dk_{2}$ \\
Lastly, we assumed from the very beginning that $d|m$, which can be expressed as $\exists k_3 \in \mathbb{Z}, m = dk_{3}$\\
From part one of the proof, we know that $n = qm + n \mathbin{\%} m$. This can now be rewritten as $n = qdk_3 + dk_2$. \\
$n = d(qk_3+k_2)$. Now, we know that $qk_3+k_2$ will produce some integer because $q, k_3, k_2$ are all defined to be integers. Let $(qk_3+k_2)$ = $k_1$. Now, we can write that $n = dk_1$, which is what we wanted to originally show.\\  

Therefore, since we've proven $d|n \Rightarrow d|n \mathbin{\%} m$ and $d|n \mathbin{\%} m \Rightarrow d|n$, we can conclude that the biconditional conclusion of the original statement is true, thus proving the whole original statement to be true. 


\end{proof}

\item[4.]

if n $\mathbin{\%}$ m is equal to zero, then we know that the greatest common divisor between n and m must be m because numbers greater than m will not be a common divisor of m due to the following property: $\forall n, d\in \mathbb{Z}^{+}, d|n \Rightarrow d \le n$. This essentially states that for some number d to divide some number n, d must be lesser than or equal to n. So, we can conclude that numbers greater than m cannot divide m and therefore cannot be considered a potential gcd between two integers m and n. Any integers lower than m that may be a common divisor between m and n are not helpful as we are looking for the greatest common divisor. 


In terms of the range for possible\_divisors, we define it to be range(1, r+1). This can be explained using the following statement: $\forall n, m, d\in \mathbb{Z}, d|m \land m \ne 0 \Rightarrow (d|n \iff d|n \mathbin{\%} m)$. What this tells us is that for some integer d to be a possible common divisor of both m and n, d must also divide r, which is defined to be $n \mathbin{\%} m$. Thus, we know that d must be less than or equal r from the following property: $\forall n, d\in \mathbb{Z}^{+}, d|r \Rightarrow d \le r$. Therefore, possible\_divisors will be in range(1, r+1). 

\begin{verbatim}
def gcd(n: int, m: int) -> int:
    """Return the greatest common divisor of m and n.

    Preconditions:
    - 1 <= m <= n
    """
    r = n % m

    if r == 0:
        return m 
    else:
        possible_divisors = range(1, r+1)
        common_divisors = {d for d in possible_divisors if divides(d, n) and divides(d, m)}
        return max(common_divisors)
\end{verbatim}
\end{enumerate}


\section*{Part 3: Wordle!}

Complete this part in the provided \texttt{a2\_part3.py} starter file.
Do \textbf{not} include your solutions in this file.

\end{document}
